
\documentclass{resume} % Use the custom resume.cls style

\usepackage[left=0.75in,top=0.6in,right=0.75in,bottom=0.6in]{geometry} % Document margins
\newcommand{\tab}[1]{\hspace{.2667\textwidth}\rlap{#1}}
\newcommand{\itab}[1]{\hspace{0em}\rlap{#1}}
\name{Seth Sawant} 
% \address{99 North Main Street \\ Cranbury, New Jersey \\  61801} 
\address{seths2@illinois.edu \\ 609-422-7817} 

\begin{document}

%----------------------------------------------------------------------------------------
%	EDUCATION SECTION
%----------------------------------------------------------------------------------------

\begin{rSection}{Education}

{\bf University of Illinois at Urbana-Champaign} \hfill {\em Fall 2017 - Present} 
\\ Bachelor of Science, Computer Engineering \hfill {GPA: 3.45/4.00}
\\ Expected Graduation: Spring 2021
\\ College of Engineering Dean's List Fall 2017, Spring 2020


{\bf Relevant Coursework}
\\ Computer Systems and Programming, Distributed Systems, Digital and Analog Signal Processing, Algorithms, Applied Parallel Programming, Computer Security, Technical Writing 
\end{rSection}


\begin{rSection}{Technical Skills}

\begin{tabular}{ @{} >{\bfseries}l @{\hspace{6ex}} l }
Launguages \& Skills &  C, C++, Python, x86, CUDA, Embedded Systems, Operating Systems\\
Software \& Tools & UNIX, Git, Arduino, Wireshark, Ghidra, Fusion 360\\
\end{tabular}

\end{rSection}

%----------------------------------------------------------------------------------------
%	WORK EXPERIENCE SECTION
%----------------------------------------------------------------------------------------

\begin{rSection}{Experience}

\begin{rSubsection}{Qualcomm}{May 2020 - August 2020}{Software Engineering Intern, Modem Common Services}{Boulder, CO}
\item Created a Qt-based log debugging tool for Qualcomm's Smart Transmit technology using Python
\item Leveraged existing Qualcomm APIs to automate binary payload unpacking and PyQt to create high-performance interactive plots, making it easy for users to quickly identify abnormal data in logs
\item Collaborated closely with end-user engineers to shape functionality of application to best increase utility and increase debugging efficiency; noted for speed of feature request turnaround
\end{rSubsection}

\begin{rSubsection}{U.S. Army Combat Capabilities Development Command}{May 2019 - August 2019}{C5ISR Night Vision and Electronic Sensors Directorate Intern}{Fort Belvoir, VA}
\item Worked with infrared sensor systems of Army's Next Generation Combat Vehicle prototype platforms 
\item Led team of contracted software engineers in time-critical development of a real time image compression pipeline using C++ and OpenCV, expediting important functionality milestones
\item Developed Python-based GUI applications to analyze and log UDP-based sensor activity and system status, drastically decreasing debugging and configuration time of systems in a field setting
\item Took initiative by authoring a hardware servicing guide for flawed media conversion equipment, praised by coworkers for both clarity and quality of documentation
\end{rSubsection}

\begin{rSubsection}{Association of Computing Machinery - SIGPwny}{August 2018 - March 2019}{Offensive Security Group Team Member}{Champaign, IL}
\item Coordinated with University faculty and staff to execute a penetration testing engagement on one of the Computer Science
department's largest undergraduate courses
\item Used knowledge of web exploitation and network attacks to help assess security of various course-critical
systems such as homework autograders, class forums, and attendance monitors
\item Learned about intricacies of conducting active information gathering on a live target
\end{rSubsection}

%------------------------------------------------

\end{rSection}

\end{document}